
The \multiexec\ tool can be used to simulate a circuit multiple
times, with different configurations. Currently the waveform generation
settings and some gate settings can be changed (\lstinline|T_P|, 
\lstinline|channel_location|, channel specific parameters). The tool also 
allows to simulate the same
circuit multiple times with the same configuration but with different
randomly generated waveforms.

\subsubsection{Configuration}\label{sec:man-multi-configuration}

The configuration of the multi\_exec tool is contained in two different
files that can be seen in Listing~\ref{lst:multiexeccfg} and ~\ref{lst:multiexecjson} 

\myinputminted[linenos,tabsize=2,breaklines,frame=single]{bash}{Configuration 
for the \multiexec\ tool.}{multiexeccfg}{code/multi_exec.cfg}

\begin{itemize}
\item
  \lstinline|MULTI_EXEC|: the flag has to be set to 1, it is used on various
  occasions in the sub-makefiles.
\item
  \lstinline|ME_CIRCUIT_DIR|: path to the circuit directory containing all the
  different circuits that can be simulated.
\item
  \lstinline|ME_CIRCUIT_UNDER_TEST|: path to the circuit that should be
  simulated.
\item
  \lstinline|ME_CONFIG_FILE|: contains information about the different
  configurations that should be simulated, more information below.
\item
  \lstinline|PRINT_LEVEL|: can be useful for reducing the output of the 
  Involution
  Tool to a minimum, otherwise the output can be quite verbose.
\end{itemize}

\myinputminted[linenos,tabsize=2,breaklines,frame=single]{js}{Configuration for 
simulation runs of the \multiexec\ tool.}{multiexecjson}{code/multi_exec.json}

\begin{itemize}
\item
  \lstinline|N|: the number of simulations that should be made with a certain
  configuration.
\item
  \lstinline|keep_waveform|: If possible, keep the waveform between two 
  simulation runs. If between two runs only parameters change which do not
  influence the waveform generation (\lstinline|T_P|, 
  \lstinline|channel_location_list|), the
  waveform can be kept. Therefore the tool always sweeps through those
  waveform independent parameters first, and then through parameters
  which affect the waveform. If \lstinline|keep_waveform| is true, the results 
  of  two simulation runs can be better compared, because otherwise two
  simulation runs would have a different input. Enabling this option
  also saves time, because the part which does not change (\spice\
  Simulation, \modelsim\ delay model simulation) is not re-executed, as it leads
  to the same result as the previous execution.
\item
  \lstinline|gate_generation|: \lstinline|channel_location_list| contains the 
  different channel locations over which should be swept,   
  (\lstinline|t_p_list|) contains the values for the pure delay $T_P$ in 
  \si{ns}.
  These settings are used for all gates, so if multiple kinds 
  of gates are used in the circuit, all gates get the same parameters. It is 
  also possible to sweep over different channel types   
  (\lstinline|channel_type_list|) and different channel parameters.
\item
  \lstinline|waveform_generation|: Contains a list of waveform generation
  configurations (see Section~\ref{sec:man-configuration-waveform}).
  If parameters are not specified here, the value of the circuit
  configuration file is used as fallback value.
\end{itemize}

\subsubsection{Execution}\label{sec:man-multi-execution}

The multi\_exec tool can be executed with a Makefile. Currently the
Makefile contains four recipes (\emph{all}, \emph{sim}, \emph{report},
\emph{clear}). \emph{all} executes the simulation (\emph{sim}) and
afterwards the multi report is generated (\emph{report}). For better
understanding which configurations are generated by the tool, the tool
saves the configurations in the temp folder under the name
\file{generate.jsonnum} and \file{gate\_config.jsonnum}. This 
way, the user can check if the configuration is as expected.

The reports of the different runs can be found in the results folder of
the circuit, in a separate folder, where all reports of this execution
are located.

\subsubsection{Multi Reporting}\label{sec:man-multi-multi-reporting}

The Involution Tool is also able to generate a report which summarizes
the results of the simulation runs. The first step of the reporting tool
is to combine the information of the  \resultsjson\ files. The
configuration for the reporting is made in the \file{multi\_exec.cfg} file as 
shown in Listing~\ref{lst:multiexecreportcfg}

\myinputminted[linenos,tabsize=2,breaklines,frame=single]{bash}{Configuration 
for the \multiexec\ reporting.}{multiexecreportcfg}{code/multi_exec_report.cfg}

\begin{itemize}
\item
  \lstinline|ME_COPY_PROPERTIES|: properties which stay the same throughout the
  execution can be specified here and are copied into the \resultsjson\ file of 
  the multi\_report.
\item
  \lstinline|ME_CALC_PROPERTIES|: Numeric properties, which should be aggregated
  can be specified here. The reporting tool calculates the minimum /
  maximum / average value of the properties over all simulation runs.
\end{itemize}

For further processing of the data, the reporting tool also provides a
CSV export, which can be configured as shown in 
Listing~\ref{lst:multiexeccsvcfg}

\myinputminted[linenos,tabsize=2,breaklines,frame=single]{bash}{Configuration 
for CSV export during \multiexec\ 
reporting.}{multiexeccsvcfg}{code/multi_exec_csv.cfg}

\begin{itemize}
\item
  \lstinline|ME_CSV_PROPERTY_ORDER|: Defines the order of the columns,
  important columns can be placed at the beginning. If this property is not 
  specified, all properties are added in an alphabetical order. Default: ""
\item
  \lstinline|ME_CSV_EXPORT_ALL_PROPERTIES|: True / False: adds all properties
  which are not specified in \lstinline|ME_CSV_PROPERTY_ORDER| at the end in
  alphabetical order. Default: True
\item
  \lstinline|ME_CSV_ESCAPE_EQUAL_SIGN|: True / False: Escapes the equal sign 
  "=", required for
  Excel, otherwise for example =-verbose shows an error, because it is
  interpreted as a formula. Default: False
\end{itemize}

The layout of the summary report can also be configured. The template
for the report is placed in \file{report\_multi.tex}. The report can consist of
several sections: 
\begin{itemize}
\item Basic information: Similar to the basic information
section of the single  report. 
\item Configurations: All configurations used are listed in this section, and 
also all simulation runs which use that simulation are linked here. By clicking 
on one of the folder names, the corresponding single report is opened (NOTE: 
Does not work with all types of pdf-readers). 
\item Power consumption: Contains a table for the
average power consumption and also tables showing the minimum / maximum
/ average deviation of the power consumption compared to the \spice\
simulation result (\spice\ / \spice) and to the result of the simulation of the
\spice\ trace with the same tool (\spice\ / column). 
\item Waveform comparison: Contains a table comparing the maximum relative and 
maximum absolute deviation for the transition count. Also shows the area of the 
deviation trace. 
\item Rankings: In this section, rankings for certain properties can
be generated. Each table shows the specified value for each simulation
run in an ordered way. This helps comparing the results of different
configurations and different waveforms. Probably the CSV export is an
easier way to do this. For which properties a ranking table should be
created can be specified in the variable \lstinline|ME_RANKING_PROPERTIES| in
the \file{multi\_exec.cfg} file.
\end{itemize}
