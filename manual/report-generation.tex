The report generation of the \invt\ attempts to be very
flexible. The main parts of the report generation are: 
\begin{itemize}
\item Scripts for extracting data from the reports of the various tools 
(\hspice, \spectre, \dc, \primetime). 
\item \LaTeX\ templates for displaying the data in a customer-defined way. 
\item Scripts for generating \texfile\ files, based on \LaTeX\ templates. 
Especially important for data that is depending on the configuration of the 
circuit and the circuit itself and has no fixed length and is recurring. 
\item Plots
\end{itemize}

\subsubsection{Scripts for extracting data}
\label{sec:man-report-scripts-for-extracting-data}

With various scripts, the data of the simulation is extracted from the
reports of the tools (\hspice, \spectre, \dc, \primetime,) and
stored in a dictionary. Unfortunately, some of the tools do not print
the data in a parser-friendly way. Therefore it is quite difficult to
write stable, version independent parsers for the reports. It can
happen, that the user has to adapt these scripts, for different versions
of the tools, especially the output format for the \dc\ and \primetime\ report 
are difficult to parse. For each file, a prefix is specified per default. 
This avoids conflicts in the dictionary containing all the parsed information. 
It also helps the user to find out how certain information from the tool report 
is named in the data
dictionary from the \invt. The extracted data is saved in the
results folder of the specific circuit in the file \resultsjson.
This file is then converted to a file called \emph{variables.tex}, which
is included by the top level report file. All the specified variables
can be used in the report files.

\subsubsection{Latex templates}
\label{sec:man-report-latex-templates}

The \invt\ supplies basic latex templates, which can be easily
adapted to fit the users needs. The templates are located in the
experiment\_setup/tex/.

\begin{itemize}
\item
  \file{report\_single.tex}: top level file which includes all the
  generated \file{*.tex} files. Basic information which needs no additional
  processing can also be included in this file. There are some
  placeholders which can be used for specifying that a certain file
  should be included here.

  \begin{itemize}
  \item
  	\lstinline|%##VARIABLES##%|
  \item
  \lstinline|%##CWG##%|
  \item
  \lstinline|%##PLOT##%|
  \item
  \lstinline|%##WAVEFORM##%|
  \item
  \lstinline|%##SCHEMATIC##%|
  \end{itemize}
\item
  \emph{cwg.tex} / \emph{cwg\_group.tex}: These two files are used as template 
  for
  displaying information about the waveform generation (see Section~\ref{sec:man-configuration-waveform}).
  \emph{cwg.tex} contains general information and can use variables from
  the data dictionary. The placeholder \emph{\%\#\#GROUPS\#\#\%} can be
  used for specifying that the information about the group configuration
  should be inserted here. The configuration for one single group is in
  the file \emph{cwg\_group.tex}. There are some placeholders which can
  be used in this file:

  \begin{itemize}
  \item
  	\lstinline|%##SIGNALS##%|: displays the signals which are in this group.
  \item
    \lstinline|%##SIGMA##%|, \lstinline|%##MUE##%|: The parameters used for 
    randomly calculating the delay for a following transition that is     
    caused by the initial transition on one of the signals of the group.
  \item
    \lstinline|%##ONEWAY##%|: displays a plain bool value if the group is a
    one-way group or not.
  \item
    \lstinline|%##ONEWAYCHECKBOX##%|: create a checked or unchecked checkbox
    which indicates if the group is a one-way group or not.
  \end{itemize}
\item
  \file{figure\_group\_template.tex} / \file{figure\_template.tex}: These two 
  files   specify how the defined figures in the config file should be
  displayed. \file{figure\_group\_template.tex} contains the layout for
  one row of figures. It can be useful to place multiple figures in
  one row, and this file specifies how this is done. The placeholder
  \lstinline|%##FIGURE##%| indicates that the \LaTeX\ code for one figure is
  inserted here. The placeholder \lstinline|%##GROUPCAPTION##%| can be used
  for adding captions to the row. If there are multiple rows, all rows
  except the last one get a \emph{phantomcaption}. After the last row of
  figures the caption of all rows is inserted. We use phantomcaption
  here, because we only want one caption after all specified figures,
  and this package ensures that the numbering of the figures is correct.
\item
  \file{waveform.tex}: Template for the table that shows the transition count
  of the different simulations for each signal. \lstinline|%##LINES##%| can be 
  used for indicating that the rows of the table should be inserted here.
\item
  \file{schematic.tex}: Specifies how the schematic of the circuit is
  displayed. The placeholder \lstinline|%##SCHEMATIC_PATH##%| can be used to
  specify the path to the image of the schematic.
\end{itemize}

\subsubsection{Report config file}\label{sec:man-report-report-config-file}

The \emph{report.cfg} shown in Listing~\ref{lst:report} file has to be placed 
in the top level of each circuit. The following code snippet shows an example 
configuration:

\myinputminted[linenos,tabsize=2,breaklines,frame=single]{bash}{Configuration for the reporting.}{report}{code/report.cfg}

Since the different tools report the power consumption in different
power units, it can be specified how the information should be displayed
in the report. The power unit in which the data is displayed in the
reports is parsed, and the extracted power information is converted to the
specified power unit. With \lstinline|FIGURES|, the figures which should be
displayed can be specified. If a schematic for the circuit should be
added to the report, the path to the schematic can be specified with
\lstinline|SCHEMATIC_PATH|. For more information on how to create the
schematic of a circuit, see Section~\ref{sec:man-howto-schematic}.

\subsubsection{Plots}\label{sec:man-report-plots}

During the report generation a number of plots is generated, based on
the result traces of \spice\ and \modelsim. In the \configcfg\ (either in
the general, or in the circuit specific config file) a number of
parameters can be set. An example is shown in Listing~\ref{lst:plots}:

\myinputminted[linenos,tabsize=2,breaklines,frame=single]{bash}{Figure plotting configuration.}{plots}{code/plots.cfg}

\begin{itemize}
\item
  \lstinline|REPORT_CONFIG|: Path to the \file{report.cfg} file.
\item
  \lstinline|FIGURE_ZOOM_NUMBER|: Configures the number of zoom plots that 
  should be generated. If the number specified is smaller than two, no zoom
  plots are generated.
\item
  \lstinline|FIGURE_ZOOM_OVERLAPPING|: Configures how much two adjacent zoom 
  plots should overlap.
\end{itemize}

The following two environment variables, shown in Listing~\ref{lst:devexport} 
specify, whether a \csvfile\ file with information about the deviation trace 
should be generated during the figure generation process. These files can be 
useful when analyzing the deviation trace in detail. The \csvfile\ files can be 
found in the same folder as the figures.

\myinputminted[linenos,tabsize=2,breaklines,frame=single]{bash}{Deviation trace export configuration.}{devexport}{code/devexport.cfg}
