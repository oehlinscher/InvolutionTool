Listing~\ref{lst:generate} shows an example configuration for the waveform
generation:

\myinputminted[linenos,tabsize=2,breaklines,frame=single]{js}{Example configuration for waveform generation.}{generate}{code/generate.json}

\subsubsection{Parameters}\label{sec:man-wave-param}

\begin{itemize}
\item
  N: the overall number of input transitions to generate. Default value: 
  \SI{100} transitions.
\item
  calc\_next\_transition\_mode:

  \begin{itemize}
  \item
    LOCAL: The randomly generated transition time is added to the time
    of the last transition on the current signal.
  \item
    GLOBAL: The randomly generated transition time is added to the time
    of the last transition on any input signal. This is the default value.
  \end{itemize}
\item
  mue / sigma: Used for parametrizing the normal distribution, which is
  responsible for generation of the time for the next transition (delta time
  since the last transition). The values are in \si{\ns}, default values: 
  \SI{0.029}{\ns} / \SI{0.010}{\ns}.
\item
  signals: A list of signals for which input waveforms should be
  generated.
  
\item rise\_time: Specifies the rise and fall time of the generated transitions 
in \si{\ns}. Default value: \SI{0.001}{\ns}. Increasing this value can be 
useful if the \spice\ simulation tool encounters numerical issues. 
  
\item
  groups: Groups can be used for specifying correlations between two or
  more signals. Groups can be useful for example, if two input signals are applied
  at the inputs of a gate. Since we want to generate short pulses, it
  can be useful to generate transitions at these input signals which are
  somehow correlated.

  \begin{itemize}
  \item
    mue / sigma: These two values are used for parametrizing the random
    number generation, which is used for calculating the transition time
    of the following transition. The random numbers are distributed
    according to a normal distribution. Values are in \si{\ns}, default values: 
    \SI{0.010}{\ns} / \SI{0.030}{\ns}.
  \item
    signals: Contains a list of two or more signals. If a transition
    happens on any signal in the list, each other signal in the list is
    checked if a following transition should be generated.
  \item
    correlation\_possibility: This parameter specifies the possibility
    that a \emph{causing} transition causes a \emph{following} transition on
    any of the other signals. This value is used for randomly deciding
    for each signal if a following transition should be generated. Default 
    value: 0.5.
  \item
    one\_way: If only signal A should cause transitions on the signals
    B, C, \ldots{} but not vice versa, this option can be set to true.
    In general, this option can be useful if one input signal (A) is
    delayed (maybe because A is first inverted), and the other input
    signal (B) is not delayed, and these two signals are applied at a
    gate. Then it can be useful that only transitions on signal A cause
    transitions on signal B. Default value: False.
  \end{itemize}
\end{itemize}
