During the development and especially during the usage of the \invt\, we came 
up with ideas for useful improvements. 
The following is a list of open tasks, each with a short description, which 
would further increase the usability and / or the performance

\begin{itemize}
	\item When \hspice\ is used, the \crossingsjson\ file is 
	generated from the \trfile\ file. \hspice\ automatically supports the 
	generation of a \vcdfile\ file, which is already the digitized version of 
	the trace. Using the \vcdfile\ for the generation of the \crossingsjson\ 
	would increase the performance. (PERFORMANCE).
	
	\item Manually implemented gates cannot be used in combination with the 
	\multiexec\ tool. The idea is to implement different architectures for 
	each combination of channel type and channel location and set the 
	different architectures during the simulations. The configuration for 
	setting the different architectures is not working yet. This has to be 
	implemented as soon as manually implemented gates are used. (FEATURE).
	
	\item When using multiple gates, the report generation for the gate 
	configuration section is not working yet. The configuration is only 
	displayed for one gate. (USABILITY).
	
	\item In \file{makeVCD.py} the symbol pool is currently fixed to a specific 
	amount of signals. For larger circuits, this has to be adapted to generate 
	allowed symbol out of a pool of valid characters. (FEATURES).
	
	\item The tool uses Python2.6 and Python3.6 (only for the \file{make.vcd}, 
	which uses \file{rawread.py}). Either change \file{rawread.py} so that it 
	can be run by Python2.6, or update all the other scripts so that they are 
	running with Python3.6 (preferred). (USABILITY).
	
	\item The name of the Makefile recipe for the simulation with the digital 
	delay model is currently named \emph{modelsim}. Since the \invt\ is not 
	only limited to \modelsim as digital simulation tool, this recipe and the 
	output folder should be renamed to something more general (\emph{make 
	digital}?). (CONSISTENCY).
	
	\item The file (\file{evaluation.csv}) that is used by the plotting tool 
	(Section~\ref{sec:man-plots}) is currently prepared manually out of the 
	\file{results.csv}. This process should be automated in the future with a 
	Python script (group data, add additional information, calculate metrics). 
	(FEATURE).
	
	\item Rethink the naming conventions for the extracted values out of the 
	different files of the tools. Especially important if different tools than 
	the standard tools are used. (USABILITY).
	
	\item The tools described in Section~\ref{sec:man-sdf-extract} and 
	Section~\ref{sec:man-generate-matching} need to be generalized, so that 
	they work for multiple circuits. (FEATURE).
\end{itemize}