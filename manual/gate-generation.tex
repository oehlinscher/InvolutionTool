Basic gates can be automatically generated by the Involution Tool. The
configuration file is placed in the general circuit directory and called
\gateconfigjson. The basic configuration can be overridden by a
file with the same name which is placed in the top level of the specific
circuit.

\subsubsection{Configuration}\label{sec:man-gate-configuration}

Listing~\ref{lst:gateconfig} shows a basic configuration for a NAND gate
with two inputs and one output:

\myinputminted[linenos,tabsize=2,breaklines,frame=single]{js}{Configuration for a NAND gate with two inputs.}{gateconfig}{code/gate_config.json}

The configuration file is a dictionary, where the key is the name of
gate, in our case \emph{ND2M1N}. Several properties can be specified: 
\begin{itemize}
\item T\_P: Specifies the pure delay in \si{\ps}, used for the 
involution channel. 
\item channel\_type: Currently \lstinline|HILL_CHANNEL| and 
\lstinline|EXP_CHANNEL| are supported. The \lstinline|EXP_CHANNEL| is the 
default setting.
\item channel\_location: The following options are available: 
\lstinline|INPUT|, \lstinline|OUTPUT|, \lstinline|OUTPUT_SWAPPED|. Either the 
involution channel is placed at the input (for each input one channel, before 
the combinatoric function) or at the output (for each output one channel). 
\lstinline|OUTPUT_SWAPPED| swaps the tr01 and tr10 times from the \sdffile\ 
file. This setting is useful for inverters. If we use the same waveform and 
compare \lstinline|INPUT| and \lstinline|OUTPUT_SWAPPED|, we receive the same 
results. This is not always the case if we compare \lstinline|INPUT| and 
\lstinline|OUTPUT|, because of the possibly asymmetric rising and falling times 
of the gate. 
\item channel\_parameters: Depending on the used channel, specific parameters 
can be specified here, which are then passed to the channel implementation. 
Currently, only \lstinline|N_UP| and \lstinline|N_DO| are supported by the 
\hillchannel. The \expchannel\ requires no additional channel parameters.
\item entity\_name:
Basically the same as the key. 
\item function: Currently only basic logic
function are supported like: \emph{and, or, nand, nor, xor, xnor, not}.
\item inputs: Specifies the names of the inputs of the gate. 
\item outputs: Specifies the name of the output. Currently, only exactly one output is
supported by the gate generation tool.
\end{itemize}


\subsubsection{Non-configurable gates}\label{sec:man-gate-non-configurable-gates}

Since the Involution Tool can only configure very basic gates, more
complex gates can be added in the folder experiment\_setup/vhdl/gates/.
These gates can then be used by all circuits. If there is a gate with
the same name in the global gate folder and the circuit specific gate
folder, the gate from the circuit specific folder is used. In order to use 
these gates with the multi execution tool, the implementation needs to 
offer an architecture for each simulated combination of 
\lstinline|channel_location| and \lstinline|channel_type|. Currently, two 
channel types and three channel locations are supported, therefore the gates 
have to offer six implementations (if all combinations should be simulated). 
Note that the switch between the architectures is not working yet (as described 
in Section~\ref{sec:man-future-development}). An example of a manually 
implemented gate can be found in experiment\_setup/vhdl/gates/ND2N1N.vhd.