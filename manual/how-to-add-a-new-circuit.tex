For adding a new circuit, various configuration files have to be added.
These files can be divided into the following two categories:

\subsubsection{Circuit files}\label{sec:man-circuit-circuit-files}

\begin{itemize}
\item
  \file{circuit.vhd}: Contains the basic structure for the testbench, required
  by ModelSim. It instantiates the unit under test, contains the signals
  and the most important thing is the placeholder   
  \lstinline|##INPUT_PROCESS##|. During the simulation process, this
  placeholder is replaced by the process which applies the generated
  waveform to the input of the circuit. For each input, one such process
  is added to the file.
\item
  \spfile / \file{*.spf}: file containing the circuit under test, defined as a 
  subcircuit, which is used in \file{main\_new.sp}. Note that for some ciruits 
  a \file{*.spf} file is used. Nevertheless, in the following sections a 
  \spice\ file has always the file extension \spfile.
\item
  \file{main\_new.sp}: This file is the main-file for the \spice\ simulation. 
  It includes the \spfile\ file containing the circuit under test.\\
  Following placeholders can be used:

  \begin{itemize}
  \item \lstinline|<VDD>|
  \item \lstinline|<VTH>|
  \item \lstinline|<TEMP>|: Can be used to set the temperature for the 
  	simulation.
  \item \lstinline|<STOPTIME>|: This is the stoptime of your
    simulation. It is set depending on the generated waveform, shortly
    after all input transitions are over.
  \item \lstinline|<signal_name>|: This placeholder is replaced
    by the generated waveform for this specific input. For each input
    you should at least specify one placeholder, so that an input is
    applied to each input port.
  \end{itemize}

  The file also contains all measurements, like the average power and
  the peak power. It is recommended to name the parameters \lstinline|pwr_avg|
  and \lstinline|pwr_max|, because the reporting utility looks for parameters
  with these names.

  Another option that can be activated is shaping. Since the input 
  signals that are generated by the waveform generation can be very steep 
  (depending on the configured \lstinline|rise_time|), and therefore not 
  realistic, one can add an inverter chain at the beginning of each input. 
  These shaping inverters use a different supply voltage, because otherwise the 
  power of the inverter chain would affect the overall power. 
  How to use shaping and the perform power measurements can be seen in the 
  following Listing~\ref{lst:main_new.sp}.
  
\myinputminted[linenos,tabsize=2,breaklines,frame=single]
{code/spice.py:SpiceLexer -x}{Example configuration for SPICE containing 
measurements and shaping.}{main_new.sp}{code/main_new.sp}
  
\item
  \file{*.v}: Contains the \verilog\ module for the circuit under test.
  Unfortunately, there is still a problem with INTERCONNECTs from the
  last gates of a circuit to the output. Therefore a \file{*.vhd}
  file can be used to specify the circuit (see \configcfg\ in the circuit 
  directory). As long as the interconnects between two gates are 0 (all our 
  test circuits had this property), using the \verilog\ file should not affect 
  the results. 
  Note that the sdf-extract tool in Section~\ref{sec:man-sdf-extract} creates 
  \sdffile\ files with INTERCONNECT = 0, since the channel model is not able to 
  incorporate INTERCONNECTs yet. Therefore the \file{*.v} file can be used 
  without problems, and the warnings during the \modelsim\ simulation can be 
  ignored.
\item
  \sdffile: Contains timing information, required for both
  delay models (\modelsim, Involution). It contains information about the
  timing regarding INTERCONNECTs and also about cell internal timing.
  
\item 
 \file{*.spef}: This file is optional, but it is highly recommended to add a 
 standard parasitics exchange format (\file{*.spef}) file to the circuit. This 
 file can be for example created during the design process with Cadence 
 Encounter, and is used during the power estimation with \dc\ and \primetime. 
 Adding such a file increases the accuracy of the power estimation in general.
\end{itemize}

\subsubsection{Configuration files}\label{sec:man-circuit-configuration-files}

\begin{itemize}
\item
  Makefile: The Makefile includes the variables from the circuit
  configuration file (\configcfg). It also forwards all commands to the
  Sub-Makefile. Most of the required variables are already set in the
  config file in the circuits directory. Probably the most important
  thing about the Makefile is that all commands from the Sub-Makefile
  can be overridden. This can be used if a circuit requires special
  simulations which cannot be done with the default set of 
  functions provided by the \invt.
\item
  \file{generate.json}, see Section~\ref{sec:man-configuration-waveform}
\item
  \file{matching}: Since the signals (especially the intermediate signals) can
  have different names between the \spfile\ file and the \file{*.v} file, we 
  need a  matching between their names. The left name is the name of the signal
  in the \spfile\ file, the right name the one from the \file{*.v} file. The
  following code snippet in Listing~\ref{lst:matching} contains an example 
  \file{matching} file:

\myinputminted[linenos,tabsize=2,breaklines,frame=single]{bash}{Example 
configuration for matching file.}{matching}{code/matching}

Note that there exists a tool for generating the \file{matching} file, see 
Section~\ref{sec:man-generate-matching}. The best way to generate the matching 
is by matching the outputs of the gates between \spfile\ and \file{*.v} file.

\item
  \file{report.cfg}: Contains configuration for the automatic report 
  generation, see Section~\ref{sec:man-configuration-report}.
\item
  \configcfg: Contains variables pointing to the configuration files and
  some other circuit specific staff which can not be specified globally
  in default \configcfg. The following code snippet in 
  Listing~\ref{lst:circuitcfg} shows an example configuration:

\myinputminted[linenos,tabsize=2,breaklines,frame=single]{bash}{Example configuration for circuit configuration file.}{circuitcfg}{code/circuit_config.cfg}

Specifying the \lstinline|TOP_DIR| is required for most of the other path
variables, since the path variables are defined relative from
\lstinline|TOP_DIR| (which is the directory of the current circuit). It is also
necessary to specify the signal names, because they are used on various
occasions. 
The \lstinline|INPUT_NAMES| contain the names of the inputs of the \verilog\ 
circuit. This variable has to be defined before including the other 
configuration files, since they make use of this variable.
The switch \lstinline|MULTI_EXEC| is necessary, because the
config files are included in a different order if the circuit is
executed from multi\_exec tool (see Section~\ref{sec:man-tools-multiexec}). 
The circuit specific file always includes the general configuration file and 
the configuration file for the used technology first. Afterwards, variables set 
in these files can be overridden, if they should be set specifically for a 
circuit. 
The \lstinline|START_OUT_NAME| is used for defining the prefix of the figure
names. The next variables are used for generating the scripts which are
executed by \modelsim, \dc\ and \primetime.
\lstinline|REQUIRED_GATES| contains a list of gates which are used by this
circuit, and should be auto generated, more information in 
Section~\ref{sec:man-configuration-gate}. 
\lstinline|SPEF_FILE_NAME| is optional, and can be used to add a standard 
parasitic exchange format (\file{*.spef}) file which is used during power 
estimation with \dc\ and \primetime. 
\lstinline|CIRCUIT_FILE_TYPE| can be either \lstinline|verilog| or 
\lstinline|vhdl|. In the first case, the \lstinline|CIRCUIT_FILE| and the 
\lstinline|VERILOG_FILE| are the same, in the latter case, 
\lstinline|CIRCUIT_FILE| is set to the \file{.vhd} file. This can be required 
when the INTERCONNECT warnings at the outputs of the circuit should be avoided.
\item \file{schematic.png}: Optional, is used for reporting (see 
Section~\ref{sec:man-howto-schematic}).
\end{itemize}
