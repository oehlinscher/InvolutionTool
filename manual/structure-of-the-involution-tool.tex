The basic structure of the involution tool is split into the following parts: 

\dirtree{%
.1 circuits/.
.2 c17\_slack\_15nm/.
.2 c17\_slack\_65nm/.
.2 inv\_tree\_15nm/.
.2 inv\_tree\_65nm/.
.2 mips\_clock\_15nm/.
.2 config.cfg.
.2 config\_15nm.cfg.
.2 config\_65nm.cfg.
.2 gate\_config.json.
.1 experiments\_setup/.
.2 python/.
.2 scripts/.
.2 spice/.
.2 tex/.
.2 vhdl/.
.2 Makefile.
.1 manual/.
%.1 manuals/.
.1 tools/.
.2 multi\_exec/.
.2 plots/.
.2 sdf\_extract/.
}

\subsubsection{Circuits}\label{sec:man-struc-circuits}

This is the folder where new circuits should be added. \cslacknm{15}, 
\cslacknm{65}, \invtreenm{15}, \invtreenm{65},  and \mipsclocknm{15} are 
five basic circuits which can be used as \emph{template} for new circuits. The 
\configcfg\ file contains variables which are used during the simulation 
process, and are used per default by all circuits, independent of the 
technology. 
These variables can be overridden in the \configcfg\ file of each circuit, 
which also includes the default \configcfgnm{xx} file, depending on the used 
technology.
More information can be found in Section~\ref{sec:man-howto-circuit}. Most of 
the variables in the \emph{config\*.cfg} files are used for defining the folder 
structure of a specific circuit and the libraries to use.

The recommended (and default) structure for a circuit is:

\begin{itemize}
\item input: contains \mainnewexp, a \spice\ file where the placeholders (for 
example for the input traces) are already replaced, and the generated waveform 
saved in a \jsonfile\ file. 
\item spice: contains the trace and logfiles of the \spice\ simulation. 
\item crossings: contains a \jsonfile\ file with all the extracted 
crossings from either the \hspice\ trace (\trfile) file or the value change 
dump (\vcdfile) file obtained with \spectre.
\item vectors: contains the prepared input trace for each input port. This 
trace is extracted from the \crossingsjson\ file and converted into a \vhdl\ 
readable version. 
\item modelsim: contains the log files, and especially the \vcdfile\ files, 
which are later used for power estimation and plotting.
\item power: contains the log files from the power estimation tools and the
generated reports. Also contains the scripts (with replaced placeholders) which 
are executed by the power estimation tools.
\item results: contains the reports for each simulation in a subfolder. Such a
subfolder contains all generated figures (showing and comparing the different 
traces and their deviation to the \spice\ reference trace), \LaTeX\ files and 
results extracted from the various result files. In case of using the 
multi\_execution tool, the reports can also be found in this directory. The 
waveform uses as input is also saved, together with the used configurations, so 
that the results can be reproduced.
\item gates: simple gates can be created automatically (more information in 
Section~\ref{sec:man-configuration-gate}). The resulting \file{*.vhd} files 
are saved in this folder.
\end{itemize}

The described folders contain the results of the different steps of the
simulation process. More information on the files required for adding a
new circuit can be found in Section~\ref{sec:man-howto-circuit}.

\subsubsection{Experiment setup:}
\label{sec:man-stuc-experiment-setup}

This folder contains a set of scripts and templates which are used
during the default simulation process. Each element of the toolchain can
be adapted to the needs of the user (by changing the Makefile of the
specific circuit, and overriding the parts of the toolchain that should
be handled differently for the circuit). The Makfile calls these scripts
in several steps. More information on the Makefile and the workflow can
be found \ref{sec:man-general-workflow}.

\begin{itemize}
\item
  python: contains python scripts used for various steps throughout the
  whole workflow (generating waveforms, parsing trace files, plotting
  figures and preparing latex reports).
\item
  scripts: scripts and templates which are used for preparing the actual
  scripts used by ModelSim and the power estimation tools.
\item
  tex: contains templates for the automatic report generation, more
  information in Section~\ref{sec:man-configuration-report}.
\item
  spice: contains useful snippets which can be added to the \spfile\ file of 
  the circuit. Currently only contains files for shaping for the \SI{15}{\nm} 
  (\shapingnm{15}) and \SI{65}{\nm} (\shapingnm{65}) technology can be used for 
  adding an inverter chain at each input.
\item
  vhdl: contains the involution delay channel implementations (\expchannel, 
  \hillchannel), templates for the testbench generation and a folder called 
  \emph{gates}, where  more complicated gates, which can not be automatically 
  generated, can be added. It also contains an implementation for a pure delay 
  channel, which is currently not used.
\end{itemize}

\begin{comment}
\subsubsection{Manuals}
\label{sec:man-stuc-experiment-manuals}

Most of the manuals in this folder are for HSPICE. Unfortunately there
are no real manuals for Design Compiler and PrimeTime, so the user has to stick 
with the man pages.
\end{comment}

\subsubsection{Manual}
\label{sec:man-stuc-experiment-manual}
Contains the \LaTeX\ sources for this manual, and also the 
resulting manual.pdf.

\subsubsection{Tools}
\label{sec:man-stuc-experiment-tools}

This folder contains tools which can be used in combination with the
Involution Tool. More information in Section~\ref{sec:man-tools}.